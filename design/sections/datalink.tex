\section{Data Link}
The foundation of the entire network stack is the data link layer which
provides reliable bidirectional packet transmission between two nodes in a shared noisy acoustic medium.
The data link layer consists of the physics sublayer and the medium access control sublayer.

\subsection{Physics Sublayer (PHY)}
Bulit on the audio I/O library, PHY enables basic data transmission with no delivery or integrity guarantee.
A PHY frame is a sequence of PCM audio signal samples that represents a chunk of bytes. It is the transmission unit on this layer.
Each frame contains a fixed preamble signal at the beginning for detection as well as synchronization and a payload signal that encodes 0/1 bits.\par
For data transmission, PHY layer constructs a frame from a chunk of bytes and send it through the medium.
For data receiving, PHY layer repeatedly pulls audio samples from the medium and seek for PHY frame based on the frequency-domain feature of the preamble.
When a frame is identified, the payload signal is extracted and decoded.
\subsubsection{Preamble Signal Design}
A chirp is a signal whose instaneous frequency increases or decreases with time.
$x_p(t)$ is a linear chirp signal whose lowest and highest instaneous frequency are $f_a$ and $f_b$ respectively.
\[
	x_p(t) = \begin{cases}
		\pi \dfrac{f_b-f_a}{T} t^2       + 2\pi f_a t     & t\in [0,T]  \\
		\pi \dfrac{f_a-f_a}{T} {(t-T)}^2 + 2\pi f_b (t-T) & t\in [T,2T]
	\end{cases}
\]
For 48kHZ sampling frequency, we choose a up-down chirp where $(f_a,f_b)=

\subsubsection{Modulation and Demodulation}
In order to transmit and receive digital signals, a modulation scheme has to be implemented.
\par
For wireless scenario, strong background noisy is detected from 0Hz to 6000Hz,
so a passband modulation scheme
combining binary phase shift keying (BPSK) and orthogonal frequency-division multiplexing (OFDM)
is implemented.
BPSK encodes bits with phase changes:
\begin{align*}
	x_0(t) & =\sin(2\pi f\, t)
	x_1(t) & =\sin(2\pi f\, t + \pi) = -\sin(2\pi f)
\end{align*}
are used to represent 0 bit and one 1 respectively.
The physics layer sends a bit by pushing samples of the modulated signal to audio I/O driver.
To recover a bit from audio signal, physics layer calculates the dot product of the receive samples and the samples of $x_0(t)$.
\[
	x_0(t)\cdot x_0(t) > 0
	\quad
	x_0(t)\cdot x_1(t) < 0
\]
A zero bit is produced if the dot product is positive, otherwise a one bit is produced.

As for cabel-connected scenario, baseband transmission is possible as low frequency background noise has less power compared with wireless scenario.



\subsubsection{Frame Detection}

\subsection{Medium Access Control sublayer}
the medium access control layer regulates access to the shared medium and implements error detection and retransmission mechanism.
