\section{Network Layer}
Network layer moves one steps further by enabling data transmission between Athernet and Internet.

\subsection{Internet Protocol}
On each Athernet node, a daemon process call the IP server constantly receives IP packets targetted to the node and forwards IP packets to peers.
Processes running on Athernet nodes uses IP accessor which communicates with IP server through datagram oriented unix domain socket to access the network layer services.\par


IP server maintains a send queue of IP packets and a reassemble queue of MAC frames.
It periodically takes one packet out of the queue, split it pieces, wrap them in MAC frames and send them to peers via MAC layer service.
When a MAC frame is received from peer, IP server append it to the reassemble queue and try to combine the received pieces into an complete IP packet.\par

IP server maintains a set of bind rules where
a rule is a 3-tuple of a transport layer protocol, a port number and a unix domain socket path.
When an IP packet is received, IP server run the following procedure:
\begin{enumerate}
	\item Inspect the IP destination field.
	      If the IP destination address is not equal to the IP address of this node, push it to the send queue.
	      Otherwise, it is a packet targeting the current Athernet node.
	\item For a packet targeting at current ndoe, extract the next level protocol field and parse the next-level destination port number.
	\item Try to find a rule that matches the protocol and port number.
	      If not such rule exists, discard the IP packet.
		  Otherwise, extract the socket path field and send the entire IP packet to the IP accessor which is bind to the socket path.
\end{enumerate}\par
An IP accessor can add or remove a bind rule for the IP server by sending bind or unbind request messages through unix domain socket.
A process can uses IP accessor to send a send packet request messages through unix domain socket to command the IP server to push a packet to the send queue.
A process will receive IP packets from the IP server that matches the bind rule.

\subsection{Network address translation}
A gateway node that is connected to the Internet uses NAT to link Athernet with the Internet.
It uses IP raw socket to capture all incoming IP packets from the OS TCP/IP stack.\par
NAT translate a 3-tuple of Athernet IP address, transport protocol and port number to another port number.
If Athernet internal node sends an IP packet targeting at an Internet host to the gateway,
the gateway node changes the IP source address to gateway IP address and the next level port number to NAT translated port number.
After recalculating the checksum, the gateway node forwards the packet out with IP raw socket.
When the gateway receives an IP packet, it goes through the reverse direction and forwards the translated IP packet to Athernet peers.\par
By doing so, Athernet internal nodes can communicate with any Internet host.
